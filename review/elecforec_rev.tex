% Options for packages loaded elsewhere
\PassOptionsToPackage{unicode}{hyperref}
\PassOptionsToPackage{hyphens}{url}
\PassOptionsToPackage{dvipsnames,svgnames,x11names}{xcolor}
%
\documentclass[
  12pt]{article}

\usepackage{amsmath,amssymb}
\usepackage{lmodern}
\usepackage{iftex}
\ifPDFTeX
  \usepackage[T1]{fontenc}
  \usepackage[utf8]{inputenc}
  \usepackage{textcomp} % provide euro and other symbols
\else % if luatex or xetex
  \usepackage{unicode-math}
  \defaultfontfeatures{Scale=MatchLowercase}
  \defaultfontfeatures[\rmfamily]{Ligatures=TeX,Scale=1}
\fi
% Use upquote if available, for straight quotes in verbatim environments
\IfFileExists{upquote.sty}{\usepackage{upquote}}{}
\IfFileExists{microtype.sty}{% use microtype if available
  \usepackage[]{microtype}
  \UseMicrotypeSet[protrusion]{basicmath} % disable protrusion for tt fonts
}{}
\makeatletter
\@ifundefined{KOMAClassName}{% if non-KOMA class
  \IfFileExists{parskip.sty}{%
    \usepackage{parskip}
  }{% else
    \setlength{\parindent}{0pt}
    \setlength{\parskip}{6pt plus 2pt minus 1pt}}
}{% if KOMA class
  \KOMAoptions{parskip=half}}
\makeatother
\usepackage{xcolor}
\setlength{\emergencystretch}{3em} % prevent overfull lines
\setcounter{secnumdepth}{5}
% Make \paragraph and \subparagraph free-standing
\ifx\paragraph\undefined\else
  \let\oldparagraph\paragraph
  \renewcommand{\paragraph}[1]{\oldparagraph{#1}\mbox{}}
\fi
\ifx\subparagraph\undefined\else
  \let\oldsubparagraph\subparagraph
  \renewcommand{\subparagraph}[1]{\oldsubparagraph{#1}\mbox{}}
\fi


\providecommand{\tightlist}{%
  \setlength{\itemsep}{0pt}\setlength{\parskip}{0pt}}\usepackage{longtable,booktabs,array}
\usepackage{calc} % for calculating minipage widths
% Correct order of tables after \paragraph or \subparagraph
\usepackage{etoolbox}
\makeatletter
\patchcmd\longtable{\par}{\if@noskipsec\mbox{}\fi\par}{}{}
\makeatother
% Allow footnotes in longtable head/foot
\IfFileExists{footnotehyper.sty}{\usepackage{footnotehyper}}{\usepackage{footnote}}
\makesavenoteenv{longtable}
\usepackage{graphicx}
\makeatletter
\def\maxwidth{\ifdim\Gin@nat@width>\linewidth\linewidth\else\Gin@nat@width\fi}
\def\maxheight{\ifdim\Gin@nat@height>\textheight\textheight\else\Gin@nat@height\fi}
\makeatother
% Scale images if necessary, so that they will not overflow the page
% margins by default, and it is still possible to overwrite the defaults
% using explicit options in \includegraphics[width, height, ...]{}
\setkeys{Gin}{width=\maxwidth,height=\maxheight,keepaspectratio}
% Set default figure placement to htbp
\makeatletter
\def\fps@figure{htbp}
\makeatother

\addtolength{\oddsidemargin}{-.5in}%
\addtolength{\evensidemargin}{-1in}%
\addtolength{\textwidth}{1in}%
\addtolength{\textheight}{1.7in}%
\addtolength{\topmargin}{-1in}%
\makeatletter
\makeatother
\makeatletter
\makeatother
\makeatletter
\@ifpackageloaded{caption}{}{\usepackage{caption}}
\AtBeginDocument{%
\ifdefined\contentsname
  \renewcommand*\contentsname{Table of contents}
\else
  \newcommand\contentsname{Table of contents}
\fi
\ifdefined\listfigurename
  \renewcommand*\listfigurename{List of Figures}
\else
  \newcommand\listfigurename{List of Figures}
\fi
\ifdefined\listtablename
  \renewcommand*\listtablename{List of Tables}
\else
  \newcommand\listtablename{List of Tables}
\fi
\ifdefined\figurename
  \renewcommand*\figurename{Figure}
\else
  \newcommand\figurename{Figure}
\fi
\ifdefined\tablename
  \renewcommand*\tablename{Table}
\else
  \newcommand\tablename{Table}
\fi
}
\@ifpackageloaded{float}{}{\usepackage{float}}
\floatstyle{ruled}
\@ifundefined{c@chapter}{\newfloat{codelisting}{h}{lop}}{\newfloat{codelisting}{h}{lop}[chapter]}
\floatname{codelisting}{Listing}
\newcommand*\listoflistings{\listof{codelisting}{List of Listings}}
\makeatother
\makeatletter
\@ifpackageloaded{caption}{}{\usepackage{caption}}
\@ifpackageloaded{subcaption}{}{\usepackage{subcaption}}
\makeatother
\makeatletter
\@ifpackageloaded{tcolorbox}{}{\usepackage[many]{tcolorbox}}
\makeatother
\makeatletter
\@ifundefined{shadecolor}{\definecolor{shadecolor}{rgb}{.97, .97, .97}}
\makeatother
\makeatletter
\makeatother
\ifLuaTeX
  \usepackage{selnolig}  % disable illegal ligatures
\fi
\usepackage[]{natbib}
\bibliographystyle{agsm}
\IfFileExists{bookmark.sty}{\usepackage{bookmark}}{\usepackage{hyperref}}
\IfFileExists{xurl.sty}{\usepackage{xurl}}{} % add URL line breaks if available
\urlstyle{same} % disable monospaced font for URLs
\hypersetup{
  pdftitle={Dynamic Forecasting of US Elections},
  pdfauthor={Marco Zanotti},
  pdfkeywords={election forecast, bayesian modelling, polls, web data},
  colorlinks=true,
  linkcolor={blue},
  filecolor={Maroon},
  citecolor={Blue},
  urlcolor={Blue},
  pdfcreator={LaTeX via pandoc}}


\begin{document}


\def\spacingset#1{\renewcommand{\baselinestretch}%
{#1}\small\normalsize} \spacingset{1}


%%%%%%%%%%%%%%%%%%%%%%%%%%%%%%%%%%%%%%%%%%%%%%%%%%%%%%%%%%%%%%%%%%%%%%%%%%%%%%

\date{June 19, 2023}
\title{\bf Dynamic Forecasting of US Elections}
\author{
Marco Zanotti\\
University of Milano Bicocca\\
}
\maketitle

\bigskip
\bigskip
\begin{abstract}
The text of your abstract 200 or fewer words.
\end{abstract}

\noindent%
{\it Keywords:} election forecast, bayesian modelling, polls, web data
\vfill

\newpage
\spacingset{1.9} % DON'T change the spacing!
\ifdefined\Shaded\renewenvironment{Shaded}{\begin{tcolorbox}[borderline west={3pt}{0pt}{shadecolor}, breakable, sharp corners, interior hidden, enhanced, boxrule=0pt, frame hidden]}{\end{tcolorbox}}\fi

\hypertarget{sec-intro}{%
\section{Introduction}\label{sec-intro}}

\addtolength{\textheight}{.5in}%

Purpose Background Challenges Gaps Directions

point out any controversy in the field

Voters at least base their decisions on relatively known and measurable
variables {[}gelman 1993{]} These fundamental variables measure their
interests and include economic conditions, party identification,
proximity of the voter's ideology and issue preferences to those of the
candidates, etc. All the serious forecasting methods try to predict the
election result using some versions of the same fundamental variables to
measure economic well-being, party identification, candidate quality and
so forth.

\hypertarget{elections-may-be-hard-to-predict}{%
\subsection{Elections may be hard to
predict}\label{elections-may-be-hard-to-predict}}

Nonostante la previsione nazionale è considerata essere prevedibile per
via del fatto che il risultato è considerato essere basato su variabili
fondamentali che sono in place before the election campaign (for
instance the economic situation of the US and forte senso di
appartenenza dei cittadini americani ad uno dei due partiti)

First, close elections will always be hard to predict since in these
cases the best possible forecast will be statistically indistinguishable
from 50\%.

In primaries, low-visibility elections, and uneven campaigns, or
uninformed elections we would not expect forecasting based on
fundamental variables meas- ured before the campaign to work. The
fast-paced events during a primary campaign (such as verbal slips,
gaffes, debates, particularly good photo, opportunities, rethorical
victories, specific policy proposals, previous primary results, etc) can
make an important difference because the can affect voters' perceptions
of the candidates' positions on fundamental issues. Also, primary
election candidates often stand so close on fundamental issues that
voters are more likely to base their decision on the minor issues that
do separates the candidates.

Moreover, the inherent instability of a multi-candidate race.

Difficulty within some well-known states.

The outcome of elections with uneven campaigns would also be hard to
predict based on fundamental variables alone.

However, in the general election campaign for the president (high
information, balanced campaigns) these events are ephemeral having
little effect on the final outcome. gelman 1993

\hypertarget{mental-process-of-voters-gelman-1993}{%
\subsection{``Mental'' Process of Voters (gelman
1993)}\label{mental-process-of-voters-gelman-1993}}

A well-accepted hypothesis of voters process during election is the so
called enlightened preferences of \citet{gelman1993}. Essentially,
voters based their preferences on fundamental variables and the function
of the electoral campaign is to inform individuals about them and their
appropriate weights. Hence, individuals are not rational but use
increasing amounts of information over the campaign. At the beginning of
the campaign voters have low level of information and this is reflected
in polls answers, while the day before the election the voters have full
information. Essentially the voters information set improves over the
course of the campaign.

Based on this assumption a model aiming at forecasting the presidential
election correctly has to incorporate the process of ``voters'
enlightment'', implying that, since the values of the fundamental
variables do not change, the weigths respondents attach to these
variables have to change during the campaign, accounting for changes in
public opinion.

\addtolength{\textheight}{-.5in}%

\hypertarget{sec-body}{%
\section{Body}\label{sec-body}}

\addtolength{\textheight}{.5in}%

explain data experimental evidence: describe important results

\addtolength{\textheight}{-.5in}%

\hypertarget{data}{%
\section{Data}\label{data}}

List, differences in the availability Nation variables or state
variables

\hypertarget{economic-and-political-indicators}{%
\subsection{Economic and Political
Indicators}\label{economic-and-political-indicators}}

Economy matters! An in-party presidential candidate running in the
context of a booming economy would win a greater share of the vote than
with a slugghish economy.

Given the relevance of the topic, numerous researchers over many decades
discovered and analysed the importance of some economic variables that
strongly affect and anticipate the election results.

This data is typically available before the start of the campaign.

growths in GDP, GNP, unemployment, inflation in the last available
quarter before the elections growths in gdp (or others) by states

incumbency (usually a dummy), national and states vote of last two
elections, presidential home-state advantage, partisanship of a state
(proportion of democrats in last legislature), president approval rating
(officially estimasted by national agencies), this consider he's foreign
affairs, personal style, communication skills, honesty, integrity,
domestic agenda, etc distance between state and candidate ideologies,
state's religion, time-for-change (if a party has controlled the White
House for two or more terms)

Sometimes also regional political variables have been adopted to
highlight southern and northern differences mainly (used to remove
anomalies in previous elections)

Many models have been developed using only such data and predicted the
results within few percentage points. Forecasting models based on
economic and political variables measured before the start of the
campaign have performed well in the past.

PROBLEMS: - more recent economic changes are difficult to incorporate
directly to economic variables since data is usually not available yet

\hypertarget{individuals-indicators}{%
\subsection{Individuals Indicators}\label{individuals-indicators}}

party, ideology, race, sex, income, education, religion, region

\hypertarget{trial-heat-polls}{%
\subsection{Trial-Heat Polls}\label{trial-heat-polls}}

horse-race aspects, interpreting each short-term change in the public
opinion polls as a serious change in the likely fortunes of candidates

Data before 1988 are usually from Gallup, then other polling
organizations emerged and are used too.

Initially only national polls then also state level polls

One can safely merge data from the different polling organizations in
order to study trends in candidate support but not the percentage of
undecided or not responding. Gelman 1993

The polls converge to a point near the actual election outcome shortly
before the election day

Even if early polls in most election years appear to have very little to
do with the eventual outcome of the general election, much evidence
exists to conclude that survey responses are related to actual voting
process, notably the predictive accuracy of polls immediately after the
election. Hence polls are connected to observable political behaviours
and incorporates the process of updating information of individuals.

Moreover, can be used to track the evolution of preferences over time
and states.

PROBLEMS: - random sampling errors (representativeness) - response
errors - question wording - different organization produce
systematically different results (organization bias) - high variability
in the support for the Democratic and Republican candidates -
non-response bias, when the candidate is going bad selectively decide
not to answer or saying they do not vote - are affected in the events of
the campaign - data limitation (availability) for states but no more a
big issue

ONE PRO: - indirectly incorporate more recent economic changes

\hypertarget{web-conversations}{%
\subsection{Web Conversations}\label{web-conversations}}

\hypertarget{methods-models}{%
\section{Methods \& Models}\label{methods-models}}

As \citet{gel:kin:1993} pointed out, one of the problems of models based
solely on economic and political indicators was that they were based on
a single regression specification based only on previous elections.

Using trial-heat polls as literal forecast produce very poor results
too, because of all the limitations of the polls. Indeed, the accuracy
of trial-heat polls in predicting presidential elections depends to a
substantial degree on when during the election year the poll is
conducted. It is commonplace now to dismiss early polls as meaningless
(same as flipping a coin) and late polls as obvious. {[}campbel 1996{]}

Undecided or non-major party vote are usually discarded or evenly
divided between the two major parties.

\hypertarget{abramowitz-1988---1996---2008}{%
\subsection{Abramowitz 1988 - 1996 -
2008}\label{abramowitz-1988---1996---2008}}

In 1988 Abram proposed the well-known Time-for-Change model The
assumption is that a presidential election is a referendum on the
performance of the incumbent president hence how voters cast their
ballots should be strongly influenced by their evaluation of the
incumbent president's performance. Captures the strength of public
sentiment for changes in the government party (based on the hypothesis
that voters attach positive value to periodic alternation between the
two major parties).

gdp growth second quarter + approval rating of incumbet president +
length of time the incumbent president's party has controlled the White
House (time for change factor)

\hypertarget{gelman-king-1993}{%
\subsection{Gelman King 1993}\label{gelman-king-1993}}

trial-heat 2 months before + incumbency + GNP changes + state variables
(state votes last 2 elections + home advantage + partisanship of state +
ideology) + regional variables + approval rating + distance between
state and candidate ideology

\hypertarget{campbell-1996}{%
\subsection{Campbell 1996}\label{campbell-1996}}

improved trial-heat literal with trial-heat + gdp changes drawback only
national

\hypertarget{brown-chappel-1999}{%
\subsection{Brown Chappel 1999}\label{brown-chappel-1999}}

Bayesian model that uses both polls and historical data and allows poll
data to be assimilated in an optimal and timely manner to update an
earlier forecast Uses time series data over elections

only simple model and generalized with description of regressors

Found gains in using polls to augment historical regression data.
Weighted average forecasts have lower MSE than historial or trial-heat
only.

\hypertarget{sec-conc}{%
\section{Conclusion}\label{sec-conc}}

\addtolength{\textheight}{.5in}%

summarize major points point out significance of results questions that
still remain to address

web data + ensembling

By treating forecasting as a Bayesian updating problem, we are able to
produce continuously revised forecasts as new poll data are released in
the course of the campaign. Allowing to account for the process of
voters and incorporating the changing weights assigned to the
fundamental variables.

Forecasting using both historical fundamental variables and poll data
outperform those based on fundamentals or polls alone (even at the state
level)

Forecasts are usually consistently accurate in the 2 months before the
election.

\addtolength{\textheight}{-.5in}%

\hypertarget{notes}{%
\section{Notes}\label{notes}}

\citet{abr:2008} offer some guidance about key ideas about statistical
ideas. \citep{abr:2008} \citep{bro:cha:1999} \citep{cam:1996}
\citep{gel:kin:1993} \citep{loc:gel:2010} \citep{rig:2009}
\citep{lin:2013} \citep{riz:2023}

{[}Consistency comparison in fitting surrogate model in the tidal power
example.{]}\{\#fig-first width=3in\}

\hypertarget{tbl-one}{}
\begin{longtable}[]{@{}lllll@{}}
\caption{\label{tbl-one}D-optimality values for design X under five
different scenarios.}\tabularnewline
\toprule()
one & two & three & four & five \\
\midrule()
\endfirsthead
\toprule()
one & two & three & four & five \\
\midrule()
\endhead
1.23 & 3.45 & 5.00 & 1.21 & 3.41 \\
1.23 & 3.45 & 5.00 & 1.21 & 3.42 \\
1.23 & 3.45 & 5.00 & 1.21 & 3.43 \\
\bottomrule()
\end{longtable}


  \bibliography{bibliography.bib}


\end{document}
