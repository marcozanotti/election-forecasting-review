% Options for packages loaded elsewhere
\PassOptionsToPackage{unicode}{hyperref}
\PassOptionsToPackage{hyphens}{url}
\PassOptionsToPackage{dvipsnames,svgnames,x11names}{xcolor}
%
\documentclass[
  12pt]{article}

\usepackage{amsmath,amssymb}
\usepackage{lmodern}
\usepackage{iftex}
\ifPDFTeX
  \usepackage[T1]{fontenc}
  \usepackage[utf8]{inputenc}
  \usepackage{textcomp} % provide euro and other symbols
\else % if luatex or xetex
  \usepackage{unicode-math}
  \defaultfontfeatures{Scale=MatchLowercase}
  \defaultfontfeatures[\rmfamily]{Ligatures=TeX,Scale=1}
\fi
% Use upquote if available, for straight quotes in verbatim environments
\IfFileExists{upquote.sty}{\usepackage{upquote}}{}
\IfFileExists{microtype.sty}{% use microtype if available
  \usepackage[]{microtype}
  \UseMicrotypeSet[protrusion]{basicmath} % disable protrusion for tt fonts
}{}
\makeatletter
\@ifundefined{KOMAClassName}{% if non-KOMA class
  \IfFileExists{parskip.sty}{%
    \usepackage{parskip}
  }{% else
    \setlength{\parindent}{0pt}
    \setlength{\parskip}{6pt plus 2pt minus 1pt}}
}{% if KOMA class
  \KOMAoptions{parskip=half}}
\makeatother
\usepackage{xcolor}
\setlength{\emergencystretch}{3em} % prevent overfull lines
\setcounter{secnumdepth}{5}
% Make \paragraph and \subparagraph free-standing
\ifx\paragraph\undefined\else
  \let\oldparagraph\paragraph
  \renewcommand{\paragraph}[1]{\oldparagraph{#1}\mbox{}}
\fi
\ifx\subparagraph\undefined\else
  \let\oldsubparagraph\subparagraph
  \renewcommand{\subparagraph}[1]{\oldsubparagraph{#1}\mbox{}}
\fi


\providecommand{\tightlist}{%
  \setlength{\itemsep}{0pt}\setlength{\parskip}{0pt}}\usepackage{longtable,booktabs,array}
\usepackage{calc} % for calculating minipage widths
% Correct order of tables after \paragraph or \subparagraph
\usepackage{etoolbox}
\makeatletter
\patchcmd\longtable{\par}{\if@noskipsec\mbox{}\fi\par}{}{}
\makeatother
% Allow footnotes in longtable head/foot
\IfFileExists{footnotehyper.sty}{\usepackage{footnotehyper}}{\usepackage{footnote}}
\makesavenoteenv{longtable}
\usepackage{graphicx}
\makeatletter
\def\maxwidth{\ifdim\Gin@nat@width>\linewidth\linewidth\else\Gin@nat@width\fi}
\def\maxheight{\ifdim\Gin@nat@height>\textheight\textheight\else\Gin@nat@height\fi}
\makeatother
% Scale images if necessary, so that they will not overflow the page
% margins by default, and it is still possible to overwrite the defaults
% using explicit options in \includegraphics[width, height, ...]{}
\setkeys{Gin}{width=\maxwidth,height=\maxheight,keepaspectratio}
% Set default figure placement to htbp
\makeatletter
\def\fps@figure{htbp}
\makeatother

\addtolength{\oddsidemargin}{-.5in}%
\addtolength{\evensidemargin}{-1in}%
\addtolength{\textwidth}{1in}%
\addtolength{\textheight}{1.7in}%
\addtolength{\topmargin}{-1in}%
\makeatletter
\makeatother
\makeatletter
\makeatother
\makeatletter
\@ifpackageloaded{caption}{}{\usepackage{caption}}
\AtBeginDocument{%
\ifdefined\contentsname
  \renewcommand*\contentsname{Table of contents}
\else
  \newcommand\contentsname{Table of contents}
\fi
\ifdefined\listfigurename
  \renewcommand*\listfigurename{List of Figures}
\else
  \newcommand\listfigurename{List of Figures}
\fi
\ifdefined\listtablename
  \renewcommand*\listtablename{List of Tables}
\else
  \newcommand\listtablename{List of Tables}
\fi
\ifdefined\figurename
  \renewcommand*\figurename{Figure}
\else
  \newcommand\figurename{Figure}
\fi
\ifdefined\tablename
  \renewcommand*\tablename{Table}
\else
  \newcommand\tablename{Table}
\fi
}
\@ifpackageloaded{float}{}{\usepackage{float}}
\floatstyle{ruled}
\@ifundefined{c@chapter}{\newfloat{codelisting}{h}{lop}}{\newfloat{codelisting}{h}{lop}[chapter]}
\floatname{codelisting}{Listing}
\newcommand*\listoflistings{\listof{codelisting}{List of Listings}}
\makeatother
\makeatletter
\@ifpackageloaded{caption}{}{\usepackage{caption}}
\@ifpackageloaded{subcaption}{}{\usepackage{subcaption}}
\makeatother
\makeatletter
\@ifpackageloaded{tcolorbox}{}{\usepackage[many]{tcolorbox}}
\makeatother
\makeatletter
\@ifundefined{shadecolor}{\definecolor{shadecolor}{rgb}{.97, .97, .97}}
\makeatother
\makeatletter
\makeatother
\ifLuaTeX
  \usepackage{selnolig}  % disable illegal ligatures
\fi
\usepackage[]{natbib}
\bibliographystyle{agsm}
\IfFileExists{bookmark.sty}{\usepackage{bookmark}}{\usepackage{hyperref}}
\IfFileExists{xurl.sty}{\usepackage{xurl}}{} % add URL line breaks if available
\urlstyle{same} % disable monospaced font for URLs
\hypersetup{
  pdftitle={Dynamic Forecasting of US Elections},
  pdfauthor={Marco Zanotti},
  pdfkeywords={election forecast, bayesian modelling, polls},
  colorlinks=true,
  linkcolor={blue},
  filecolor={Maroon},
  citecolor={Blue},
  urlcolor={Blue},
  pdfcreator={LaTeX via pandoc}}


\begin{document}


\def\spacingset#1{\renewcommand{\baselinestretch}%
{#1}\small\normalsize} \spacingset{1}


%%%%%%%%%%%%%%%%%%%%%%%%%%%%%%%%%%%%%%%%%%%%%%%%%%%%%%%%%%%%%%%%%%%%%%%%%%%%%%

\date{June 17, 2023}
\title{\bf Dynamic Forecasting of US Elections}
\author{
Marco Zanotti\\
University of Milano Bicocca\\
}
\maketitle

\bigskip
\bigskip
\begin{abstract}
The text of your abstract. 200 or fewer words.
\end{abstract}

\noindent%
{\it Keywords:} election forecast, bayesian modelling, polls
\vfill

\newpage
\spacingset{1.9} % DON'T change the spacing!
\ifdefined\Shaded\renewenvironment{Shaded}{\begin{tcolorbox}[frame hidden, boxrule=0pt, interior hidden, borderline west={3pt}{0pt}{shadecolor}, enhanced, sharp corners, breakable]}{\end{tcolorbox}}\fi

\hypertarget{sec-intro}{%
\section{Introduction}\label{sec-intro}}

Body of paper. Margins in this document are roughly 0.75 inches all
around, letter size paper.

\hypertarget{sec-meth}{%
\section{Methods}\label{sec-meth}}

Don't take any of these section titles seriously. They're just for
illustration.

\hypertarget{sec-verify}{%
\section{Verifications}\label{sec-verify}}

This section will be just long enough to illustrate what a full page of
text looks like, for margins and spacing.

\addtolength{\textheight}{.5in}%

\citet{gelm:veht:2021} offer some guidance about key ideas about
statistical ideas. On an unrelated note, spreadsheets are important to
use correctly \citep{brom:woo:2018}. Log-linear models are an attractive
way to model categorical data \citep{bish:fien:1975}.

The quick brown fox jumped over the lazy dog. The quick brown fox jumped
over the lazy dog. The quick brown fox jumped over the lazy dog. The
quick brown fox jumped over the lazy dog. \textbf{With this spacing we
have 25 lines per page.} The quick brown fox jumped over the lazy dog.
The quick brown fox jumped over the lazy dog. The quick brown fox jumped
over the lazy dog. The quick brown fox jumped over the lazy dog. The
quick brown fox jumped over the lazy dog.

The quick brown fox jumped over the lazy dog. The quick brown fox jumped
over the lazy dog. The quick brown fox jumped over the lazy dog. The
quick brown fox jumped over the lazy dog. The quick brown fox jumped
over the lazy dog. The quick brown fox jumped over the lazy dog. The
quick brown fox jumped over the lazy dog. The quick brown fox jumped
over the lazy dog. The quick brown fox jumped over the lazy dog. The
quick brown fox jumped over the lazy dog.

The quick brown fox jumped over the lazy dog. The quick brown fox jumped
over the lazy dog. The quick brown fox jumped over the lazy dog. The
quick brown fox jumped over the lazy dog. The quick brown fox jumped
over the lazy dog. The quick brown fox jumped over the lazy dog. The
quick brown fox jumped over the lazy dog. The quick brown fox jumped
over the lazy dog. The quick brown fox jumped over the lazy dog. The
quick brown fox jumped over the lazy dog.

The quick brown fox jumped over the lazy dog. The quick brown fox jumped
over the lazy dog. The quick brown fox jumped over the lazy dog. The
quick brown fox jumped over the lazy dog. The quick brown fox jumped
over the lazy dog. The quick brown fox jumped over the lazy dog. The
quick brown fox jumped over the lazy dog. The quick brown fox jumped
over the lazy dog. The quick brown fox jumped over the lazy dog. The
quick brown fox jumped over the lazy dog.

\addtolength{\textheight}{-.5in}%

\hypertarget{sec-conc}{%
\section{Conclusion}\label{sec-conc}}

Here is the conclusion.

\hypertarget{supplementary-material}{}
\bigskip

\begin{center}

{\large\bf SUPPLEMENTARY MATERIAL}

\end{center}

{[}Consistency comparison in fitting surrogate model in the tidal power
example.{]}\{\#fig-first width=3in\}

\hypertarget{tbl-one}{}
\begin{longtable}[]{@{}lllll@{}}
\caption{\label{tbl-one}D-optimality values for design X under five
different scenarios.}\tabularnewline
\toprule()
one & two & three & four & five \\
\midrule()
\endfirsthead
\toprule()
one & two & three & four & five \\
\midrule()
\endhead
1.23 & 3.45 & 5.00 & 1.21 & 3.41 \\
1.23 & 3.45 & 5.00 & 1.21 & 3.42 \\
1.23 & 3.45 & 5.00 & 1.21 & 3.43 \\
\bottomrule()
\end{longtable}


  \bibliography{bibliography.bib}


\end{document}
